\section{Ergebnisse und Interpretation}
In diesem Abschnitt werden die Ergebnisse der im Abschnitt Versuchsaufbau und Durchführung beschriebenen Abgabe präsentiert und interpretiert.
\subsection{Abgabe 1}
\subsubsection{Aufgabe 5 (a)}
\begin{figure}[H]
    \centering
    \includegraphics[width=\textwidth]{figures/Lab1_IMU_6Achsentest_Accelerometer.png} 
    \caption{Abgabe 1 - Aufgabe 5 (a)}
    \label{fig:6Achsentest_Accelerometer}
\end{figure}
In Abbildung \ref{fig:6Achsentest_Accelerometer} ist die gemessene Beschleunigung des 6-Achsen-Tests dargestellt.
Die X-, Y- und Z-Achse sind farblich unterschiedlich gekennzeichnet.
Es ist zu erkennen, dass die Beschleunigung in den drei Achsen variiert,
wenn der Sensor in verschiedene Positionen gebracht wird. Dies bestätigt die Funktionsfähigkeit des Beschleunigungssensors.
\newline
Die Farben der entsprechenden Achsen sind wie folgt:
\begin{itemize}
    \item X-Achse: Rot
    \item Y-Achse: Grün
    \item Z-Achse: Blau
\end{itemize}


\begin{figure}[H]
    \centering
    \includegraphics[width=\textwidth]{figures/Lab1_IMU_6Achsentest_Accelerometer_Filtered.png} 
    \caption{Abgabe 1 - Aufgabe 6}
    \label{fig:6Achsentest_Filtered}
\end{figure}
In der obigen Abbildung sind die gleichen Achsen wie in Abbildung \ref{fig:6Achsentest_Accelerometer} verwendet worden.
Der eingesetzte Butterworth-Filter 4. Ordnung hat erreicht, dass die hochfrequenten Anteile des Signals deutlich reduziert wurden.
Dadurch ist das Signal deutlich geglättet und Rauschen wurde entfernt. Da dieser Filter eine hohe Flankensteilheit besitzt,
werden die niederfrequenten Anteile des Signals kaum beeinflusst, was zu einer guten Signalqualität führt.

\subsubsection{Aufgabe 8}
An dieser Stelle muss angemerkt werden, dass die in Abbildung \ref{fig:10s-1Achse_Accelerometer} dargestellten Daten nach dem Anpassen der falsch eingestellten Messparameter aufgezeichnet wurden.
Da wir keine falschen Messwerte aus Aufgabe 8 haben, werden wir uns auf die Beantwortung der Fragen aus den Aufgaben 9 und 12 konzentrieren.

\subsubsection{Aufgabe 9 und 12}

\begin{figure}[H]
    \centering
    \includegraphics[width=0.8\textwidth]{figures/Lab1_10s-1Achse_3Anlauf_Accelerometer.png} 
    \caption{Abgabe 1 - Aufgabe 9}
    \label{fig:10s-1Achse_Accelerometer}
\end{figure}
In Abbildung \ref{fig:10s-1Achse_Accelerometer} ist die gemessene Beschleunigung des 10 Sekunden Tests auf einer Achse dargestellt.
Die X-, Y- und Z-Achse sind farblich wie folgt gekennzeichnet:
\begin{itemize}
    \item X-Achse: Blau
    \item Y-Achse: Orange
    \item Z-Achse: Grün
\end{itemize}
Es ist zu erkennen, dass die meiste Beschleunigung in der X-Achse auftritt, was darauf hinweist, dass die Bewegung hauptsächlich entlang dieser Achse stattgefunden hat.
Die Y- und Z-Achsen zeigen nur geringe Schwankungen, was darauf hindeutet, dass die Bewegung in diesen Richtungen minimal war.
\newline
Folgende Anpassungen mussten vorgenommen werden, um die Messdaten korrekt aufzuzeichnen:

\begin{itemize}
    \item Die Baudrate wurde auf 57600 Baud (Bits per second) geändert, damit die Schnittstelle und der Arduino/Sparkfun die gleiche Baudrate haben.
    \item setscale wurde von 2G auf 8G geändert, um die höheren Beschleunigungswerte korrekt zu erfassen und kein Clipping entsteht (Plateaus).
    \item Die Sample-Rate wurde von 1,56 Hz auf 800 Hz (\texttt{ODR\_1} zu \texttt{ODR\_800}) erhöht, um dem Nyquist-Shannon-Abtasttheorem gerecht zu werden und Aliasing-Effekte zu vermeiden. 
    Bei einer Abtastrate von 800 Hz beträgt die Nyquist-Frequenz 400 Hz, was für die Erfassung der relevanten Signalkomponenten bei unserer Beschleunigungsmessung ausreichend ist.
\end{itemize}
Die Informationen zu den Anpassungen wurden dem Datenblatt des Sensors auf Seite 1 entnommen \cite{NXP2023MMA8452Q}.
Mögliche Parameter für die Auswahl der setscale sind 2G, 4G und 8G.
Um Plateaus zu vermeiden, sollte der höchste Wert (8G) gewählt werden, um Clipping zu vermeiden.

\subsubsection{Aufgabe 10}
Aus den aufgezeichneten Daten der Abbildung \ref{fig:10s-1Achse_Accelerometer} wurde die Berechnung der Messfrequenz durchgeführt.
Zu unserer Überraschung ergab die Berechnung eine Abtastfrequenz von ca. 270 Hz, obwohl die Sample-Rate auf 800 Hz eingestellt war.
Dies dürfte vor allem auf die eingestellte Baudrate von 57600 Baud zurückzuführen sein,
da die Übertragung der Daten über die serielle Schnittstelle begrenzt ist.

\subsubsection{Aufgabe 13}
Diese Aufgabe konnte leider nicht durchgeführt werden, da das Aufzeichnen von Messdaten mit Energieversogung per 9V Blockbatterie nicht funktionierte.
Auch die Verwendung des Reset-Buttons führte nicht zum Erfolg.
\newline
Welcher Unterschied besteht zwischen der Aufzeichnung mit und ohne Batterieversorgung?
Die maximale Messfrequenz ist durch die Schreibgeschwindigkeit der SD-Karte begrenzt.
Beim Auslesen der Daten über die serielle Schnittstelle ist die Übertragungsrate (Baudrate) der limitierende Faktor,
dieser ist jedoch deutlich höher als die Schreibgeschwindigkeit der SD-Karte.