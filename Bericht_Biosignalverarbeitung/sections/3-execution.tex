\section{Versuchsaufbau und Durchführung}
Im Folgenden Abschnitt wird der Versuchsaufbau beschrieben und die Durchführung des Experiments erläutert.
\subsection{Versuchsaufbau}
Die zum Versuchsaufbau verwendeten Geräte und Materialien sind im Abschnitt Vorbereitende Arbeiten aufgelistet.

\begin{figure}[H]
    \centering
    \includegraphics[width=1\textwidth]{figures/Lab1_Versuchsaufbau.jpg} 
    \caption{Versuchsaufbau}
    \label{fig:versuchsaufbau}
\end{figure}
Wie in Abbildung \ref{fig:versuchsaufbau} zu sehen ist, wurde der Mikrocontroller über das USB-Kabel mit dem Laptop und über ein Qwiic-Kabel mit dem Beschleunigungssenors verbunden.

\subsection{Durchführung}
\subsubsection{Aufgabe 1 und 2 - Teil 1-4}
Aufgabe 1 und 2 (Teil 1-4) wurden bereits im Abschnitt Vorbereitende Arbeiten beschrieben.
\subsubsection{Aufgabe 2 - Teil  5-6}
Mithilfe der Analyse aus Punkt 4, konnte Punkt 5 (Koordinatensystem) beantwortet werden.
Anschließend wurden die Daten die im Serial Monitor angezeigt und per copy-paste in ein txt-Datei gespeichert (Punkt 6).
\subsubsection{Aufgabe 3}
Zuerst wurde, wie in Punkt 1 beschrieben, die Funktion des Dataloggers überprüft, indem das Testprogramm ausgeführt wurde und anschließen die Daten im PC-Interface betrachtet wurden. Hier gab es zuerst Schwierigkeiten, die aber nach einigen Versuchen überwunden werden konnten.
Daraufhin wurde die Verbindung des Computers mit dem Sparkfun Board getrennt und die Stromversorgung über die 9V Batterie hergestellt (Punkt 2). Aufgrund von massiven Problemen die trotz troubleshooting nicht behoben werden konnten wurde auf Anweisung des Lektors die Aufgabe abgebrochen und zur nächsten weitergegangen.
\subsubsection{Aufgabe 4}
In der Aufgabe 4 wurden die Daten der mikro-SD-Karte des Dataloggers ausgelesen und im bereitgestellten Jupyter-Notebook ausgewertet.

\subsection {Abgabe 1}
\begin{minipage}{\textwidth}
In der Abgabe wurden folgende Punkte erledigt:
\begin{itemize}
    \item 5 (a)
    \item 6
    \item 8 (a)
    \item 9
    \item 10
    \item 12
\end{itemize}
\end{minipage}
\newline

Die Aufgaben wurden entsprechend der Anweisungen erledigt und im Jupyter-Notebook dokumentiert.

Aufgabe 13 wurde nicht erledigt, da die Datenaufzeichnung mit dem Datalogger per 9V Batterie nicht funktionierte.
