\section{Vorbereitende Arbeiten}
Die Aufgaben 1 und 2 sind als vorbereitende Maßnahmen bearbeitet worden.
Im Rahmen der Aufgabe 1 wurden der Treiber für den Arduino Mikrokontroller installiert und der Blinktest durchgeführt.
Die zweite Aufgabe beinhaltete das Einbinden der Sparkfun spezifischen Bibliotheken für Board und Sensor in die Arduino IDE.
Zudem wurde mit einem Beipsielskript erstmals Daten ausgelesen, über den seriellen Plotter ausgegeben und mit dem seriellen Monitor gespeichert.

\subsection{Verwendete Hardware}
Die Laboreinheit wurde mit den folgenden Hardware-Komponenten durchgeführt:

\begin{itemize}
    \item Fixe Komponenten für alle Aufgaben
        \begin{itemize}
            \item Mikrocontroller (Sparkfun)
            \item Micro-USB-Kabel
            \item Beschleunigungssenor (MMA8452Q)
            \item Verbindungskabel (Qwiic)
        \end{itemize}
    \item benötigte Komponenten für mobiles Setup
        \begin{itemize}
            \item 9V Blockbatterie
            \item 9V Batterieanschluss
            \item Datalogger (Qwiic Open Log)
            \item Micro-SD Karte
            \item Verbingunskabel (Qwiic)
        \end{itemize}
    \item Verwendete Software
        \begin{itemize}
            \item Arduino IDE
            \item Python 3.13.9 für unser Python Visualisierungskript
            \item .ino Skript Lab1Code1.ino für die Datenerfassung
        \end{itemize}
\end{itemize}

Für die verschiedenen Aufgaben wurden unterschiedliche Kombinationen der oben genannten Hardware-Komponenten verwendet.
Bei entsprechenden Aufgaben werden die Hardware Kombinationen und der genaue Versuchsaufbau näher beschrieben.