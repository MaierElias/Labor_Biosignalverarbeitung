\section{Einleitung und Zielsetzung}
Die Laboreinheiten haben das Ziel den Studierenden die Grundlagen der Biosignalverarbeitung näherzubringen.
Darin soll ein Crashkurs in den Umgang mit Arduino IDE, C und Arduino Uno enthalten sein, sowie in die Datenaquise mit dem Mikrocontroller-Setup.
Damit gehen der Aufbau der Hard- und Software zur Detektion von Bewegungen und die Programmierung zur Analyse von Beschleunigungsdaten einher.

Im Labor 1 lag der Fokus auf dem Erkennen der richtigen Parameter für die Aufnahme der Daten eines IMU's, Inertial Measurement Unit, und dem Verwenden eines externen Dataloggers.
Dabei wurden hauptsächlich ein Arduino Mikrocontroller vom Typ CH340 und ein MMA8452Q IMU-Sensor verwendet. In einer kleineren Teilaufgabe sollte auch ein Qwiic Open Log Datalogger angeschlossen werden, um die aufgezeichneten Daten zu speichern.
Allerdings sollte die Teilaufgabe mit Datalogger auf Grund von Zeitmangel nur kurz behandelt werden, sodass diese im Laborbericht 1 nicht weiter beschrieben wird.
Zur Aufnahme der Daten wurde die Bibliothek Sparkfun benutzt, welche eine einfache Schnittstelle zur Kommunikation mit dem Sensor bietet.