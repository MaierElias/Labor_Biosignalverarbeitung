\section{Versuchsaufbau und Durchführung}

Die Verstärkerschaltung, die gemäß der Simulation aufgebaut wurde, befindet sich auf einem Steckbrett. Die Stromversorgung für den Versuch wurde über ein externes Netzteil mit einer Spannung von $V_{CC} = \SI{24}{\volt}$ bereitgestellt. Der Funktionsgenerator speiste ein sinusförmiges Eingangssignal mit einer variablen Frequenz im Bereich von \SI{100}{\hertz} bis \SI{10}{\mega\hertz} in die Schaltung ein. Die gemessene Ausgangsspannung wurde kontinuierlich aufgezeichnet, um die Verstärkung des Verstärkers und den Frequenzgang zu analysieren.

\subsection*{Aufbau des Labors}

Zu Beginn des Versuchs wurde der Laboraufbau gemäß den Vorgaben aus der Laborvorgabe ~\cite{lab-guide} realisiert. Der Versuch wurde unter Verwendung der im Labor verfügbaren Geräte durchgeführt, die eine präzise Messung der Ausgangsspannung ermöglichten. Die folgende Abbildung zeigt den Gesamtaufbau des Labors mit den verwendeten Messgeräten.

\subsection*{Geräteübersicht}

Im Rahmen des Versuchs wurden folgende Geräte verwendet:

\begin{table}[H]
\centering
\begin{tabular}{ll}
\toprule
\textbf{Gerät} & \textbf{Verwendung} \\
\midrule
Funktionsgenerator & Signalquelle für Eingangssignale (Sinus, Rechteck) \\
Oszilloskop & Messung der Ausgangsspannung und Analyse des Frequenzgangs \\
Steckbrett & Aufbau der Verstärkerschaltung \\
BNC-Tastköpfe & Signalabgriff am Eingang und Ausgang des Verstärkers \\
Multimeter & Messung der Versorgungsspannung und Stromstärke \\
\bottomrule
\end{tabular}
\caption{Verwendete Geräte im Laborversuch}
\label{tab:geraete_lab3}
\end{table}

\subsection*{Verstärkerschaltung}

Die Verstärkerschaltung, die als Ausgangspunkt für die Messungen diente, wurde auf einem Steckbrett aufgebaut. Der verwendete Verstärker war eine klassische OPV-Schaltung, deren Eigenschaften mit Hilfe einer LTSpice-Simulation vorab überprüft wurden. Die folgende Abbildung zeigt die schematische Darstellung der Emitterverstärkerschaltung, die im Experiment verwendet wurde.

% \begin{figure}[H]
%     \centering
%     \includegraphics[width=0.5\linewidth]{images/Ermitterverstärkerschaltung.jpg} 
%     \caption{Emitterverstärkerschaltung für den Versuch}
%     \label{fig:emitterschaltung}
% \end{figure}

\subsection*{Versuchsablauf}

Der Ablauf des Experiments war wie folgt:
\begin{enumerate}
  \item Aufbau der Verstärkerschaltung auf dem Steckbrett gemäß den Vorgaben ~\cite{lab-guide}, Simulationsergebnissen und Berechnungen.
  \item Verbindung des Funktionsgenerators und Oszilloskops mit der Schaltung über BNC-Tastköpfe und Splitter, um das Eingangssignal und die Ausgangsspannung zu visualisieren.
  \item Einspeisung sinusförmiger Signale mit variablen Amplituden, von \SI{1}{\milli\volt_{pp}} bis \SI{1}{\volt_{pp}}.
  \item Dokumentation der Ausgangsamplitude bei verschiedenen Frequenzen, um die Verstärkungscharakteristik des Verstärkers zu überprüfen.
  \item Aufzeichnung des Frequenzgangs und Erstellung des Bode-Diagramms basierend auf den experimentellen Daten.
\end{enumerate}
