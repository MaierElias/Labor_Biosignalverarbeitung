\section{Vorbereitende Arbeiten}
Im Voraus zur Laborübung wurden der Treiber für den Arduino Mikrokontroller installiert und nach Sakaianleitung erste Tests durchgeführt,
bei denen der Umgang mit IMU, dem seriellen Monitor sowie Plotter ausprobiert wurde und erste Daten gespeichert werden konnten.
Zudem wurde als vorbereitende Maßnahme das Python-Skript zur Visualisierung der vom IMU gewonnenen CSV-Dateien durchgearbeitet und mit genannten Testdateien getestet.

Die Laboreinheit wurde mit den folgenden Hardware-Komponenten durchgeführt:
\begin{itemize}
    \item Mikrocontroller (Sparkfun)
    \item Analog Digital Converter
    \item Data-Logger
    \item microSD-Karte
    \item Beschleunigungssenor (IMU)
    \item 9V Blockbatterie
    \item 9V Batterieanschluss
    \item Micro-USB-Kabel
    \item Qwiic Kabel
    \item Jumper Kabel
\end{itemize}