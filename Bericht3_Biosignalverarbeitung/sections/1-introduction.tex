\section{Einleitung und Zielsetzung}
In diesem Labor soll die Aufzeichnung und Verarbeitung von EMG-Signalen
(EMG: Elektromyographie) erlernt werden. Dafür werden EMG-Signale von
verschieden intensiven Muskelaktivitäten aufgenommen und anschliessend mit
verschiedenen Methoden der Signalverarbeitung analysiert. Ziel ist es, die
Unterschiede der Signale zu erkennen und zu quantifizieren.
Die EMG-Signale werden mit Hilfe von Oberflächenelektroden erfasst, die auf der
Hautoberfläche über dem Muskel angebracht werden. Die Signale werden dann
verstärkt, gefiltert und digitalisiert, um sie für die weitere Analyse
vorzubereiten.