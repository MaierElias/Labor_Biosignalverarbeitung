\section{Einleitung und Zielsetzung}
Ziel dieses Versuchs ist die Analyse elektromyographischer (EMG-)Signale zur Quantifizierung der 
Muskelaktivität sowie zur Untersuchung von Ermüdungsprozessen in der Skelettmuskulatur. Anhand von Messungen
 der maximalen willkürlichen Kontraktion (MVC), belastungsabhängiger Muskelaktivierung und einer
  Ermüdungsaufgabe sollen sowohl zeit- als auch frequenzbasierte Kenngrößen des EMG-Signals bestimmt werden. 
  Ein besonderer Fokus liegt dabei auf der normierten Darstellung der Muskelaktivität in Prozent der 
  individuellen MVC sowie auf der Veränderung spektraler Parameter als Indikator für muskuläre Ermüdung.
\newline
Zu Beginn werden die aufgezeichneten EMG-Rohdaten vorverarbeitet, um Störanteile und Artefakte zu eliminieren.
 Dies umfasst die Offset-Korrektur, eine bandpassgefilterte Signalaufbereitung, die Gleichrichtung sowie die 
 Berechnung der Einhüllenden des EMG-Signals. Anschließend wird aus mehreren MVC-Durchläufen eine gemittelte 
 maximale Muskelaktivierung bestimmt, die als Referenz für alle weiteren Auswertungen dient. Darauf aufbauend
wird die relative Muskelaktivität während statischer Belastungen und unter Ermüdung berechnet. Abschließend
erfolgt eine Frequenzanalyse der EMG-Signale mittels spektraler Leistungsdichte, wobei insbesondere die 
Medianfrequenz zu verschiedenen Zeitpunkten der Muskelaktivierung bestimmt wird, um Veränderungen der 
Kontraktionsgeschwindigkeit der Muskelfasern im Verlauf der Ermüdung zu untersuchen.