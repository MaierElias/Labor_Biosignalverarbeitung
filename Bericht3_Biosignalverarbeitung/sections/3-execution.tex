\section{Versuchsaufbau und Durchführung}
\subsection{Aufgabe 1}

\subsection{Aufgabe 2}
\subsection{Aufgabe 3}
\subsection{Aufgabe 4: Aufbau des MVC-Versuchsaufbaus}
Hier Bild von MVC Aufbau einfügen!!
\newline
Wie im Bild zu erkennen ist, besteht der MVC-Versuchsaufbau aus mehreren Komponenten. Es wurden drei Elektroden, vergleichbare zu denen, 
die schon im Lab 2 verwendet wurden, am Probanten angebracht. Eine auf dem Bauch des Bizeps Brachii, eine zwei Zentimeter weiter in Richtung
der Sehne und eine Referenzelektrode auf dem C7 Halswirbel. Die Platzierung der Groundreferenzelektrode auf dem C7-Wirbel wurde gewählt, um
Störungen oder andere Artefakte durch Bewegung zu minimieren.
\newline
Der Probant sitzt auf einem Stuhl, sodass der Unterarm auf dem Oberschenkel aufliegt und sich mit kleinster Bewegung ein 90°-Winkel zwischen
Ober- und Unterarm bildet. Das Handgelenk liegt dann an der Unterkante des Tisches an. Um die MVC zu bestimmen, wird der Probant angewiesen,
so stark wie möglich den Tisch anzuheben, während eine weitere Person auf dem Tisch sitzt, um zu gewährleisten, dass der sich sich nicht
bewegt, obwohl die maximale Kraft vom Probanten aufgebracht wird.
\newline
Der Probant hält diese maximale Anspannung für acht bis zehn Sekunden, während die EMG-Daten aufgezeichnet werden. Dieser Vorgang wird
drei Mal wiederholt, mit einer Pause von etwa einer Minute zwischen den Versuchen, um Muskelermüdung zu vermeiden. Der MVC-Versuch wurde für
alle drei Gruppenmitglieder durchgeführt.
\subsection{Aufgabe 5}
\subsection{Aufgabe 6}
\subsection{Aufgabe 7}
\subsection{Aufgabe 8}
\subsection{Aufgabe 9}