\section{Versuchsaufbau und Durchführung}
\subsection{Aufgabe 1: Diagramme der Komponenten}

\subsection{Aufgabe 2: Daten im Seriellen Plotter}
Im seriellen Plotter wurden die Rohdaten des EKG-Signals visualisiert. Dabei konnte beobachtet werden,
dass, sobald der Laptop über ein Netzteil mit Strom versorgt wurde, starke Störungen in Form von Noise im Signal auftraten.
Der Noise hat eine Frequenz von circa 50 Hz, weshalb davon auszugehen ist, dass es sich um Netzbrummen handelt.
Das Netzbrummen wurde durch die Verwendung des Laptops im Akkubetrieb vermieden. Alternativ wäre eine Filterung des Signals mit Tiefpassfiltern möglich gewesen.

\subsection{Aufgabe 3: Experiment in Ruhe}

\subsection{Aufgabe 4: Beschreibung und Erklärung des Ruhe-EKG Codes}
\subsubsection{Arduino Code}
Das bereitgestellte Skript Lab2Code1 \cite{Lab2Code1.ino} wurde verwendet, um die Rohdaten des EKG-Signals zu erfassen und über die serielle Schnittstelle an den Computer zu übertragen.
Von dort aus wurden die Daten im nachfolgenden Skript serialRead \cite{serialRead.ipynb} eingelesen und gespeichert, wie in Abschnitt \ref{sssec:PythonCode} beschrieben wird.
Der Code initialisiert die serielle Kommunikation mit einer Baudrate von 500000 und liest kontinuierlich die analogen Werte vom EKG-Sensor ein.
Dieser Wert wird dann nur an die Console weitergegeben, wenn eine Zeit von 1000 Millisekunden vergangen ist.
\subsubsection{Python Code}
\label{sssec:PythonCode}
Der Code serialRead \cite{serialRead.ipynb} wurde ebenfalls bereitgestellt. 
In diesem Skript können die serielle Schnittstelle, die Baudrate und die Dauer der Aufnahme sowie der Name des Outputdokuments als Variablen gesetzt werden.
Die Funktion sampling() öffnet die serielle Schnittstelle und das Outputdkoument und liest dann die Daten für die angegebene Dauer ein.
Die eingelesenen Daten werden in eine Liste gespeichert und anschließend in das Outputdokument geschrieben.
Danach werden Informationen zur gemittelten Samplingrate über die gesamte Aufnahmezeit, totale Anzahl an Samples und vergangener Zeit sowie der Speicherbestätigung ausgegeben.
Die Funktion returned den Wert der errechneten Sampling Rate.
Im zweiten Teil des Codes werden die Daten aus dem Outputdokument eingelesen und in einem Plot visualisiert, um direkt die Qualität der Aufnahme beurteilen zu können.

\subsection{Aufgabe 5: Fünf-Sekunden-Plot der Ruhe-EKGs}

\subsection{Aufgabe 6: Errechnete Daten der Ruhe-EKGs}

\subsection{Aufgabe 7: Einordnung der Daten im Kontext derer der Mitstudierenden}


\subsection{Aufgabe 8: Plott der Herzfrequenz während des Belastungs-EKGs}

\subsection{Aufgabe 9: Ruhephase vor Belastungs-EKG}

\subsection{Aufgabe 10: Erholungsphase nach Belastungs-EKG}

\subsection{Aufgabe 11: Berechnen des metabolischen Energieverbrauchs}

\subsection{Aufgabe 12: Einordnung des Energieverbrauchs und entsprechender Code}