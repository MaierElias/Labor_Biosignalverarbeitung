\section{Versuchsaufbau und Durchführung}
\subsection{Aufgabe 1: Diagramme der Komponenten}

\subsection{Aufgabe 2: Daten im Seriellen Plotter}

\begin{figure}[h]
    \centering
    \includegraphics[width=0.7\textwidth]{figures/ecg_plots_with_and_without_charger.png}
    \caption{Serieller Plotter mit (oben) und ohne (unten) Netzanschluss}
    \label{fig:serial_plotter_noise_and_normal}
\end{figure}

Im seriellen Plotter wurden die Rohdaten des EKG-Signals visualisiert. Wie in Abbildung \ref{fig:serial_plotter_noise_and_normal} abgebildet konnte beobachtet werden,
dass, sobald der Laptop über ein Netzteil mit Strom versorgt wurde, starke Störungen in Form von Noise im Signal auftraten.
Diese Störung hat eine Frequenz von circa 50 Hz, weshalb davon auszugehen ist, dass es sich um Netzbrummen handelt.
Dies wurde durch die Verwendung des Laptops im Akkubetrieb vermieden. Alternativ wäre eine Filterung des Signals mit Tiefpassfiltern möglich gewesen.

\subsection{Aufgabe 3: Experiment in Ruhe}
\begin{figure}[h]
    \centering
    \includegraphics[width=0.7\textwidth]{figures/ecg_plots_all_participants.png}
    \caption{Rohdaten der Ruhe-EKGs für alle Teilnehmenden}
    \label{fig:ecg_plot_all_participants}
\end{figure}
In Abbildung \ref{fig:ecg_plot_all_participants} sind die Rohdaten der Ruhe-EKGs für alle Teilnehmenden für den Zeitraum von ca. 5 Sekunden (5000 ms) dargestellt.
- hier muss noch genau auf die Fragestellung eingegangen werden -

\begin{figure}[h]
    \centering
    \includegraphics[width=0.7\textwidth]{figures/ecg_plot_Elias.png}
    \caption{EKG-Signal mit markierten P-, QRS- und T-Wellen für Elias}
    \label{fig:ecg_with_marked_waves_elias}
\end{figure}

In Abbildung \ref{fig:ecg_with_marked_waves_elias} ist das EKG-Signal von Elias mit den markierten P-(rot), QRS-(grün) und T-Wellen(gelb) dargestellt.
Die P-Welle repräsentiert die Depolarisation der Vorhöfe. Sie entsteht, wenn der elektrische Impuls vom Sinusknoten ausgeht und
sich über beide Vorhöfe ausbreitet. Dies führt zur Kontraktion der Vorhöfe und zum Bluttransport in die Herzkammern. Die P-Welle
ist normalerweise klein und positiv.
Der QRS-Komplex zeigt die Depolarisation der Herzkammern. Dieser scharfe, große Ausschlag entsteht, wenn sich der elektrische
Impuls vom AV-Knoten über das His-Bündel und die Purkinje-Fasern schnell durch die Ventrikel ausbreitet. Die resultierende Kontraktion der
Ventrikel pumpt das Blut in den Lungen- und Körperkreislauf. Der QRS-Komplex überlagert gleichzeitig die Repolarisation der Vorhöfe, die
dadurch im EKG nicht sichtbar ist.
Die T-Welle repräsentiert die Repolarisation der Ventrikel. Nach der Kontraktion kehren die Herzkammerzellen in ihren Ruhezustand zurück.
Dies bereitet das Herz auf den nächsten Zyklus vor. Die T-Welle ist
normalerweise positiv und breiter als der QRS-Komplex, da die Repolarisation langsamer abläuft als die Depolarisation.


\subsection{Aufgabe 4: Beschreibung und Erklärung des Ruhe-EKG Codes}
\subsubsection{Arduino Code}
Das bereitgestellte Skript Lab2Code1 \cite{Lab2Code1.ino} wurde verwendet, um die Rohdaten des EKG-Signals zu erfassen und über die serielle Schnittstelle an den Computer zu übertragen.
Von dort aus wurden die Daten im nachfolgenden Skript serialRead \cite{serialRead.ipynb} eingelesen und gespeichert, wie in Abschnitt \ref{sssec:PythonCode} beschrieben wird.
Der Code initialisiert die serielle Kommunikation mit einer Baudrate von 500000 und liest kontinuierlich die analogen Werte vom EKG-Sensor ein.
Dieser Wert wird dann nur an die Console weitergegeben, wenn eine Zeit von 1000 Millisekunden vergangen ist.
\subsubsection{Python Code}
\label{sssec:PythonCode}
Der Code serialRead \cite{serialRead.ipynb} wurde ebenfalls bereitgestellt. 
In diesem Skript können die serielle Schnittstelle, die Baudrate und die Dauer der Aufnahme sowie der Name des Outputdokuments als Variablen gesetzt werden.
Die Funktion sampling() öffnet die serielle Schnittstelle und das Outputdkoument und liest dann die Daten für die angegebene Dauer ein.
Die eingelesenen Daten werden in eine Liste gespeichert und anschließend in das Outputdokument geschrieben.
Danach werden Informationen zur gemittelten Samplingrate über die gesamte Aufnahmezeit, totale Anzahl an Samples und vergangener Zeit sowie der Speicherbestätigung ausgegeben.
Die Funktion gibt den Wert der errechneten Sampling Rate zurück.
Im zweiten Teil des Codes werden die Daten aus dem Outputdokument eingelesen und in einem Plot visualisiert, um direkt die Qualität der Aufnahme beurteilen zu können.

\subsection{Aufgabe 5: Fünf-Sekunden-Plot der Ruhe-EKGs}

\begin{figure}[h]
    \centering
    \includegraphics[width=0.7\textwidth]{figures/ecg_with_rwaves_Elias.png}
    \caption{EKG-Signal mit markierten R-Zacken für Elias}
    \label{fig:ecg_with_rwaves_elias}
\end{figure}    

Abbildung \ref{fig:ecg_with_rwaves_elias} zeigt das EKG-Signal von Elias mit den markierten R-Zacken in einem ausgesuchten Bereich mit ca. 5 Sekunden Länge.
Die Positionen der R-Zacken wurden durch die Funktion \texttt{find\_peaks()} aus dem Modul \texttt{scipy.signal} ermittelt.

\subsection{Aufgabe 6: Errechnete Daten der Ruhe-EKGs}

\subsection{Aufgabe 7: Einordnung der Daten im Kontext derer der Mitstudierenden}
Im Histogramm zur Verteilung der Herzfrequenzen \ref{fig:histogram_herzfrequenz} zeigt sich, dass die mittlere Herzfrequenz bei Frauen und Männern ähnlich verteilt ist,
wobei beide Geschlechter einen Mittelwert um 63-65 bpm aufweisen.
\begin{figure}[htbp]
    \centering
    \includegraphics[width=0.5\textwidth]{figures/Histogramm_Herzfrequenz.png}
    \caption{Histogramm der Herzfrequenzverteilung nach Geschlecht}
    \label{fig:histogram_herzfrequenz}
\end{figure}

Im Histogramm zur Herzfrequenzvariabilität \ref{fig:histogram_hfv} ist eine größere Streuung erkennbar,
wobei beide Geschlechter eine breite Verteilung zeigen.
\begin{figure}[htbp]
    \centering
    \includegraphics[width=0.5\textwidth]{figures/Histogramm_HFV.png}
    \caption{Histogramm der Herzfrequenzvariabilität nach Geschlecht}
    \label{fig:histogram_hfv}
\end{figure}

Für beide Histogramm lässt sich sagen, dass die großen Überlappungen zwischen den Geschlechtern darauf hindeuten,
dass die interindividuelle Variabilität größer ist als geschlechtsspezifische Unterschiede sind. Mögliche Ursachen für nicht erkennbare 
Geschlechtsunterschiede könnten die relativ kleine Stichprobengröße, ähnliche Fitnesslevel der Teilnehmenden oder unterschiedliche
Messbedingungen sein.
\newline
Unsere Werte für die Herzfrequenz und die Herzfrequenzvariabilität liegen im erwarteten Bereich für gesunde Erwachsene in Ruhe.
Auch im direkten Vergleich mit den Werten der Mitstudierenden zeigen sich keine Auffälligkeiten, außer der Tatsache, dass die Variabilität
der Herzfrequenzen bei einigen Anderen als auch bei Hauke höher ist als die, die normalerweise für gesunde Erwachsene erwartet werden.
Das lässt auf eine etwas ungenaue Auswertung bzw. Messungen schließen.
\newline


\subsection{Aufgabe 8: Plot der Herzfrequenz während des Belastungs-EKGs}

\subsection{Aufgabe 9: Ruhephase vor Belastungs-EKG}

\subsection{Aufgabe 10: Erholungsphase nach Belastungs-EKG}

\subsection{Aufgabe 11: Berechnen des metabolischen Energieverbrauchs}

\subsection{Aufgabe 12: Einordnung des Energieverbrauchs und entsprechender Code}