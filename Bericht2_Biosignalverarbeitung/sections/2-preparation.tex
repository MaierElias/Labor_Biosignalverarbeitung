\section{Vorbereitende Arbeiten}
Es wurden im Rahmen dieser Laboreinheit folgende Komponenten und Software verwendet: Die Tabelle muss noch auf die Richtlinie angepasst werden.

\begin{itemize}
    \item \textbf{Mikrocontroller} - Arduino Uno R3
    \item \textbf{EMG/EKG Sensor mit Elektroden} - hier vllt noch Art/Marke erwähnen
    \item \textbf{USB-Kabel} - zum Verbinden des Arduino mit dem Computer
    \item \textbf{3 Jumper Kabel} - zum Verbinden des Sensors mit dem Arduino
    \item \textbf{Arduino IDE} - Software zur Programmierung des Arduino Mikrocontrollers
    \item \textbf{Python (Jupiter Notebook / VS Code)} - Software zur Verarbeitung und Visualisierung der Sensordaten
\end{itemize}
Das Skript zum Aufzeichen und Speichern der Sensor Rawdaten wurde von Team der Lehrerenden bereitgestellt und konnte ohne große Anpassungen verwendet werden.
--> hier vllt Link zu Datei oder Repository einfügen <--
Auch für die Visualisierung der Daten wurde ein Skript zur Verfügung gestellt, welches unter Anderem auf Grund von Artefakten leicht auf unseren spezifischen Fall angepasst werden musste.
--> hier vllt Link zu Datei oder Repository einfügen <--

Vor Beginn der eigentlichen Messungen musste der Arduino Uno R3 Mikrocontroller über die drei Jumper Kabel mit dem EMG/EKG Sensor verbunden werden.
Dazu wurden die Kabel wie folgt angeschlossen: HIER BILD VON AUFBAU EINFÜGEN, BESCHRIFTUNG DIESES MAL GETIPPT UND NICHT VON HAND

Die Elektroden des Sensors wurden anschließend an den Probanden/ die Probanden angebracht. Die drei Elektroden sind am Manubrium, am linken V6 Ableitpunkt und am C7 der Halswirbelsäule angeklebt worden.
HIER MUSS AUCH NOCH EIN BILD HIN, ELIAS HAT ES GERADE NOCH