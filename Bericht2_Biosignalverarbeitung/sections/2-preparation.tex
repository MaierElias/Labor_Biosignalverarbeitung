\section{Vorbereitende Arbeiten}
Es wurden im Rahmen dieser Laboreinheit folgende Komponenten und Software verwendet:
\begin{table}[ht]
    \centering
    \begin{tabularx}{\textwidth}{l l X}
        \hline
        Typ & Modell & Verwendung \\
        \hline
        Microcontroller & Arduino Uno R3 & Zur Erfassung der Sensordaten und Übertragung an den Computer \\
        Sensor & EMG/EKG Sensor & Zur Messung der elektrischen Signale des Herzens \\
        Software & Arduino IDE 1.8.19 & Zur Programmierung des Arduino Mikrocontrollers \\
        Software & Python 3.9 (Jupyter Notebook / VS Code) & Zur Verarbeitung, Aufzeichnung und Visualisierung der Sensordaten \\
        USB-Kabel & Standard USB A to B Kabel & Zur Verbindung des Arduino mit dem Computer mittels COM4 \\
        3 Jumper Kabel & Standard Jumper Kabel & Zur Verbindung des EMG-Sensors mit dem Arduino \\
        Elektroden & Standard EKG-Elektroden & Zur Ableitung der elektrischen Signale des Herzens \\
        \hline
    \end{tabularx}
    \caption[Verwendete Komponenten und Software]{Die im Rahmen der Laboreinheit verwendeten Komponenten und Software}
    \label{tab:verwendete_komponenten}
\end{table}


Das Skript zum Aufzeichen und Speichern der Sensor Rawdaten wurde von Team der Lehrerenden bereitgestellt und konnte ohne große Anpassungen verwendet werden.
--> hier vllt Link zu Datei oder Repository einfügen <--
Auch für die Visualisierung der Daten wurde ein Skript zur Verfügung gestellt, welches unter Anderem auf Grund von Artefakten leicht auf unseren spezifischen Fall angepasst werden musste.
--> hier vllt Link zu Datei oder Repository einfügen <--

Vor Beginn der eigentlichen Messungen musste der in \ref{tab:verwendete_komponenten} erwähnte Arduino Uno R3 Mikrocontroller über die drei Jumper Kabel mit dem EMG/EKG Sensor verbunden werden.
Dazu wurden die Kabel wie folgt angeschlossen: HIER BILD VON AUFBAU EINFÜGEN, BESCHRIFTUNG DIESES MAL GETIPPT UND NICHT VON HAND

Die Elektroden des Sensors wurden anschließend an den Probanden/ die Probanden angebracht. Die drei Elektroden sind am Manubrium, am linken V6 Ableitpunkt und am C7 der Halswirbelsäule angeklebt worden.
HIER MUSS AUCH NOCH EIN BILD HIN, ELIAS HAT ES GERADE NOCH